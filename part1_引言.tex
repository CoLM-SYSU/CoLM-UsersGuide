\part{引言}
\section{引言}

% 通用陆面模式2024版(The Common Land Model 2024, CoLM 2024)是通用陆面模式的第四个版本(即CLM 1.0, CoLM 2004, CoLM 2014, CoLM 2024)。它是集模式、数据集、性能评估和高性能计算为一体的、自成体系且功能完备的陆面综合模拟研究平台,可广泛应用于数值天气预报/气候预测、水文水资源、生态环境、城市、农林牧等行业的科学研究和精细化业务,适用于多尺度(\textasciitilde 1米至\textasciitilde 100千米)应用。

通用陆面模式版本2024 (The Common Land Model version 2024, CoLM 2024)是一集模式、数据集、性能评估和高性能计算为一体的自成体系的陆面综合模拟研究平台, 可广泛应用于数值天气预报/气候预测、水文水资源、生态环境、城市、农林牧等行业的科学研究和精细化业务,适用于多尺度(~1米至~100千米)应用。无论您是从事天气/气候/地球系统等研究与应用工作,还是水文与水资源、生态环境、以及人类活动影响等研究与应用工作,CoLM 2024将为您的工作提供强大的工具,助您高质量高效率完成与陆面过程相关的工作。

CoLM 2024版有以下主要改进或新增,

1. \textbf{新增多种网格和次网格划分方法}。除常用的经纬度网格外,CoLM 2024版还提供了另外两种网格,一种是可根据陆表异质性进行任意加密的非结构网格,另一种是适用于山坡尺度陆面过程模拟的流域单元网格。在网格单元内部,CoLM 2024版可根据地表覆盖类型、植被功能型或者植物群落对被自然植被覆盖的地表进行进一步的次网格划分。

2. \textbf{使用了更为丰富的基础数据集}。CoLM 2024版对土地覆盖/土地利用数据、土壤数据、植被结构及属性数据、水文数据、城市数据、作物数据和离线大气驱动数据做了全面的更新或补充。基于新的数据处理和尺度转换方法,在模式中将多源的、具有不同分辨率的数据融合到一起使用。

3. \textbf{改进或新增多个水文和能量过程的参数化方案}。基于新的网格和次网格结构以及基础数据,CoLM 2024版建立了基于三维植被的辐射传输、湍流、叶片温度和植被截留降水方案,发展了基于物理原理的产流和汇流方案,并新增了积雪内辐射传输、植被水力和生物地球化学等重要过程。

4. \textbf{增加了多个人类活动相关过程}。这些过程包括以三维城市建筑群落为基本结构假设的城市模式、包含多种调度规则的水库模式、土地利用与土地覆盖变化方案以及作物、火灾等模式,为陆面过程与人类活动的相互作用研究提供了有力的工具。

5. \textbf{提升了模式的易操作性和可移植性}。CoLM 2024版保持了之前版本代码易读易修改的优点,优化了模式配置和数据读写方法,同时针对高分辨率模拟设计了新的并行计算方案。

\subsection{用户指南的目标}

本用户指南的目标是为您提供全面的指导,帮助您理解和高效使用CoLM 2024。CoLM 2024的一个主要优势是其模块化设计,使得您可以根据具体需求灵活选择或调整各个功能模块,可以根据您需求进行定制化配置,确保满足不同应用场景的需求。
本用户指南旨在帮助您顺利安装、配置和使用CoLM 2024。通过本手册,您将能够快速了解如何准备输入数据、配置参数、运行、优化和调试等操作。
本指南不仅为初学者提供了详细的入门指导,还为经验丰富的用户提供了更深入的使用建议和优化方法。无论您是第一次使用CoLM 2024,还是需要解决运行过程中遇到的复杂问题,本指南都将是您必不可少的参考资料。

\subsection{手册结构}

本手册第二部分介绍了如何快速地运行一个CoLM 2024版模式实例。第三部分包含对所有模式配置的详细描述。在第四部分,通过多个例子,讲解了如何根据具体的研究题目配置模式,进行多种情形下的陆面过程模拟。第五部分对模式中的部分代码做了深入的解释,有助于模式开发者对CoLM 2024版进行修改和发展。

\subsection{支持与反馈}

尽管本用户指南已经涵盖了大多数使用案例,但由于每个用户的需求和应用背景可能有所不同,可能会在使用过程中遇到一些特定问题或挑战。我们鼓励通过邮件或在线支持平台与我们联系,反馈问题和建议。我们的技术支持团队将为用户提供及时的帮助,并根据用户反馈持续优化模式和用户指南。

\textbf{联系人及邮件地址:}

张树鹏 <zhangshp8@mail.sysu.edu.cn>

陆星\hbox{\scalebox{0.6}[1]{吉}\kern-.2em\scalebox{0.6}[1]{力}} <luxingj@mail.sysu.edu.cn>.


\clearpage
